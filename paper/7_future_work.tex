\section{Future Work}
 
\newcommentone{Malacology is a first step towards showing how general-purpose
storage systems can be adapted to target special-purpose applications.  By
encapsulating storage system functionality as reusable building blocks, we
enable application developers to leverage storage capabilities based on
interfaces that are proven and understandable.  However, creation and
composition of interfaces is complex; constructs must be combined safely in
order to provide correctness, performance and security. We will study
additional Malacology-based services in order to learn techniques that support
safe composition.}\addressesissue{5}
 
\newcommentone{Some higher-level services that we plan to build using the
interfaces in Table~\ref{table:examples} are: an elastic cloud database, a data
processing engine, and a data layout manager. Approaches proposed so far use
the data I/O interface to push down predicates and computation, the file type
interface to maintain access paths and metadata efficiently, and the durability
interface to manage ingestion and movement. Using the programmable storage
approach helps us build higher-level services that work well with the storage
system not in spite of it.}
 
\newcommentone{ Our experience with ZLog and Mantle demonstrates that the labor
of wrapping existing services in reusable interfaces is justified by the power
and flexibility that this encapsulation affords to programmers.  In exchange
for this flexibility, however, programmers may forfeit the \emph{protection
from change} afforded by narrow storage interfaces such as the POSIX API.  To
implement applications on programmable storage systems such as Malacology,
programmers must find solutions by navigating a complex design space,
simultaneously addressing functional correctness, performance and fault
tolerance. Worse still, their solutions may be sensitive to changes in the
underlying environment, such as hardware upgrades, software version changes and
evolving workloads.  For example, a major version change in Ceph required us to
rewrite significant parts of ZLog to maintain acceptable performance. Each such
evolution costs developer time and risks introducing bugs.}
 
\newcommentone{We are actively exploring the use of high-level
\emph{declarative} languages based on Datalog~\cite{alvaro:cidr11} to program
data access and storage APIs.  Using this approach, a systems programmer can
specify the functional behavior in a relational (or algebraic) language,
allowing an optimizer to search through the space of functionally equivalent
physical implementations and select a good execution plan, re-optimizing when
storage characteristics or statistics change.  Much like query planning and
optimization in database systems~\cite{hellerstein:cidr15}, this approach will
separate the concerns of correctness and performance, protecting applications
(which usually evolve slowly) against changes in more dynamic storage system.}
\addressesissue{3}


%\section{Future Work}
%
%\newcommentone{This work is one point along the path towards producing
%general-purpose storage systems that can target special-purpose applications.
%Continuing to expore existing storage system functionality as reusable building
%blocks will enable the proliferation of storage-centric APIs, allowing
%developers to custom-fit their applications to take advantage of the
%capabilities of the underlying system. When adding new interfaces and composing
%them together, careful consideration should be made to address security, access
%control, and safety. Correctness and performance will surely be affected if
%interfaces are combined poorly and these drawbacks will be quantified.}
%\addressesissue{5}
%
%\newcommentone{The next higher-level services we plan to build using the
%interfaces in Table~\ref{table:examples} are: an elastic cloud database, a data
%layout manager, and a data processing engine. Some cloud databases aggressively
%cache data to hide the overheads and weaker consistency semantics of the
%distributed file systems they sit on~\cite{pavlo:sigmod16} but using Malacology
%we could build a database that works well because of the storage system not in
%spite of it. We plan to use the data IO interface for push down predicates, the
%file type interface for maintaining indices and stats, and the durability
%interface internals for ingest and data movement. The data layout manager would
%be similar to DRepl~\cite{ionkov:msst2013} and would use the file type,
%durability, and data I/O interface to manage different views of data. Finally,
%the data processing engine would be similar to MapReduce and would use the data
%I/O, shared resource, and load balancing interface to implement the resource
%manager, computation engine, and storage platform in the Hadoop ecosystem.}
%
%\newcommentone{ Our experience with Zlog and Mantle illustrates that the labor
%of wrapping existing services in interfaces is justified by the power and
%flexibility they afford to programmers.  In exchange for this flexibility,
%however, programmable storage systems force programmers to forfeit the
%\emph{protection from change} afforded by narrow storage interfaces like the
%POSIX API.  To implement applications built atop systems such as Malacology,
%programmers must navigate a complex design space, simultaneously addressing
%often orthogonal concerns including functional correctness, performance and
%fault tolerance. Worse still, these compositions can be sensitive to changes in
%the underlying environment (such as hardware upgrades, software version changes
%and evolving workloads).  For example, a major version change in Ceph required
%us to rewrite significant parts of Zlog to maintain acceptable performance --
%each such evolution costs developer time and risks introducing bugs.}
%
%\newcommentone{We are actively exploring the use of high-level
%\emph{declarative} languages based on Datalog~\cite{alvaro:cidr11} to program
%storage APIs.  In this way, a programmer can specify the functional behavior of
%a storage interface once in a relational (or algebraic) language, allowing an
%optimizer to model search the space of functionally equivalent physical
%implementations, and re-optimizing when characteristics change.  Much like
%query planning and optimization in database systems~\cite{hellerstein:cidr15},
%this approach will separate the concerns of correctness and performance, and
%protect applications (which evolve slowly) against changes in highly dynamic
%storage system.} \addressesissue{3}
