\begin{notes}
\textcolor{red}{
\noindent Issues raised by reviewers, paper augmented as noted inline (red/blue text):
\begin{enumerate}
  \item clarify what was implemented: diagram of interfaces and prose
  specifying the difference between Malacology and prior work (reviewers B, C,
  D, E)
  \item describe what makes a programmable storage system (reviewers C, D)
  \item organize motivation: group interfaces into categories (e.g., core
  functionality, features, and performance optimizations) and describe other
  applications we can build with Malacology (reviewers B, C)
  \item clarify terminology: service vs. file system metadata, storage vs.
  file system, dirty vs. clean slate approaches, soft-state caching (reviewers
  B, C)
  \item address security, access control, and the safety; specify how it
  affects composability, performance, and multi-tenancy (reviewer D)
\end{enumerate}
}
\end{notes}

\begin{abstract} Storage systems are caught between the need to evolve data
processing applications efficiently and quickly, and the increasing velocity
with which storage device technology evolves. This puts tremendous pressure on
storage systems to support rapid change both in terms of their interfaces and
their performance. But adapting storage systems can be difficult because
unprincipled changes might jeopardize years of code-hardening and performance
optimization efforts that were necessary for users to entrust their data to the
storage system. We introduce the programmable storage approach, which exposes
internal services and abstractions of the storage stack as building blocks for
higher-level services. We also build a prototype to explore how existing
abstractions of common storage system services can be leveraged to adapt to the
needs of new data processing systems and the increasing variety of storage
devices.  We illustrate the advantages and challenges of this approach by
composing existing internal abstractions into two new higher-level services: a
\newcommentone{file system} metadata load balancer and a high-performance
distributed shared-log.  The evaluation demonstrates that our services inherit
desirable qualities of the back-end storage system, including the ability to
balance load, efficiently propagate service metadata, recover from failure, and
to navigate trade-offs between latency and throughput using leases.
\end{abstract}
