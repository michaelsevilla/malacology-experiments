\section{Conclusion}
\label{conclusion-and-future-work}

Programmable storage is a viable method for eliminating duplication of complex
error-prone software used as workarounds for storage system deficiencies. We
propose that systems expose their services in a safe way allowing application
developers to customize system behavior to meet their needs while not
sacrificing correctness. To illustrate the benefits of this approach we
presented Malacology\footnote{\url{http://programmability.us}}, a
programmable storage system that facilitates the construction of new services
by re-purposing existing subsystem abstractions of the storage stack. 

\oldcommentone{Future work will focus on constructing more interfaces to
support a wide variety of storage system services that can be configured
on-the-fly in existing systems. This work is one point along that path to
producing general-purpose storage systems that can target  special-purpose
applications.  Ultimately we want to utilize declarative methods for expressing
new services.}

\acks

We thank the EuroSys reviewers for their hard work, attentiveness, and
genuinely helpful suggestions. We especially thank Mahesh Balakrishnan for
shepherding the paper. This work was partially funded by the
Center for Research in Open Source
Software\footnote{\url{http://cross.ucsc.edu}}, the DOE Award
DE-SC0016074, and the NSF Award 1450488.

\textbf{Note}: this paper follows The Popper
Convention\footnote{\url{http://falsifiable.us}}~\cite{jimenez_popper_2016}. All 
the experiments presented here are available on the repository associated to 
this 
article\footnote{\url{https://github.com/michaelsevilla/malacology-popper/tree/v2.1}}. 
For every figure, a \texttt{[source]} link points to a Jupyter notebook that 
shows the analysis from where the graph was obtained; its parent folder contains 
all the associated artifacts.
