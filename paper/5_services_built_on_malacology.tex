\section{Services Built on Malacology}
\label{sec:services}
\label{services-built-on-malacology}

\label{services}
%\begin{table}
%\centering\small
%\begin{tabular}{  l | l | l }
%\textbf{Service}              &
%\textbf{Used to...}           &
%\textbf{Malacology Interface} \\ \hline
%Mantle & load balance metadata     & Load Balancing  \\
%       & agree on current balancer & Service Metadata\\
%       & control client write lock & Shared Resource \\
%       & persist policy logic      & Durability      \\ \hline
%ZLog   & implement CORFU           & Object \& Data  \\ 
%       & install ZLog interfaces   & Service Metadata\\
%       & persist log entries       & Durability      \\ \hline
%ZLog   & load balance sequencers   & Load Balancing  \\
%Seqr.  & manage access to log tail & Shared Resource \\ 
%\end{tabular}
%\caption{Mantle and ZLog are built using Malacology interfaces so they inherit the
%robustness of the subsystems of the underlying stack.}
%\label{table:implementation}
%\end{table}

In this section we describe two services built on top of Malacology. The first
is Mantle, a framework for dynamically specifying file system metadata load balancing
policies. The second system, ZLog, is a high-performance distributed
shared-log.  In addition to these services, we will demonstrate how we combine
ZLog and Mantle to implement service-aware metadata load balancing policies.

\subsection{Mantle: Programmable Load Balancer}
\label{sec:mantle}

Mantle~\cite{sevilla:sc15-mantle} is a programmable load balancer that separates the metadata balancing
policies from their mechanisms. Administrators inject code to change how the
metadata cluster distributes metadata. Our previous work showed how 
to use Mantle to implement a single node metadata
service, a distributed metadata service with hashing, and a distributed
metadata service with dynamic subtree partitioning. 

The original implementation was ``hard-coded" into Ceph and lacked robustness
(no versioning, durability, or policy distribution).  Re-implemented using
Malacology, Mantle now enjoys (1) the versioning provided by Ceph's monitor daemons and (2) the durability and
distribution provided by Ceph's reliable object store.  Re-using the internal
abstractions with Malacology resulted in a 2\(\times\) reduction in source code
compared to the original implementation.

\subsubsection{Versioning Balancer Policies}

Ensuring that the version of the current load balancer is consistent across the
physical servers in the metadata cluster was not addressed in the original
implementation. The user had to set the version on each individual server and
it was trivial to make the versions inconsistent. Maintaining consistent
versions is important for cooperative balancing policies, where
local decisions are made assuming properties about other instances in the cluster.

With Malacology, Mantle stores the version of the current load balancer in the
Service Metadata interface. The version of the load balancer corresponds
to an object name in the balancing policy. Using the Service Metadata interface
means Mantle inherits the consistency of Ceph's internal
monitor daemons. The user changes the version of the load balancer using a new
CLI command.

\subsubsection{Making Balancer Policies Durable}

The load balancer version described above corresponds to the name of an object
in RADOS that holds the actual Lua balancing code.  When metadata server nodes start
balancing load, they first check the latest version from the metadata server map and
compare it to the balancer they have loaded. If the version has changed, they
dereference the pointer to the balancer version by reading the corresponding
object in RADOS. This is in contrast to the original Mantle implementation which stored load
balancer code on the local file system -- a technique which is unreliable and
may result in silent corruption.

The balancer pulls the Lua code from RADOS synchronously; asynchronous reads
are not possible because of the architecture of the metadata server. The synchronous
behavior is not the default behavior for RADOS operations, so we achieve this
with a timeout: if the asynchronous read does not come back within half the
balancing tick interval the operation is canceled and a Connection Timeout
error is returned. By default, the balancing tick interval is 10 seconds, so
Mantle will use a 5 second second timeout.

This design allows Mantle to immediately return an error if anything
RADOS-related goes wrong.  We use this implementation because we do not want to
do a blocking object storage daemon read from inside the global metadata server lock. Doing so would bring
down the metadata server cluster if any of the object storage daemons are not responsive.

Storing the balancers in RADOS is simplified by the use of an interpreted
language for writing balancer code. If we used a language that needs to be
compiled, like the C++ object classes in the object storage daemon, we would need to ensure
binary compatibility, which is complicated by different operating systems,
distributions, and compilers.

\subsubsection{Logging, Debugging, and Warnings}

In the original implementation, Mantle would log all errors, warnings, and
debug messages to a log stored locally on each metadata server. To get the simplest
status messages or to debug problems, the user would have to log into each metadata server
individually, look at the logs, and reason about causality and ordering.

With Malacology, Mantle re-uses the centralized logging features of the
monitoring service. Important errors, warnings, and info messages are collected
by the monitoring subsystem and appear in the monitor cluster log so instead of
users going to each node, they can watch messages appear at the monitor daemon.
Messages are logged sparingly, so as not to overload the monitor with frivolous
debugging but important events, like balancer version changes or failed
subsystems, show up in the centralized log.

\subsection{ZLog: A Fast Distributed Shared Log}
\label{sec:zlog}

The second service implemented on Malacology is ZLog, a high-performance
distributed shared-log that is based on the CORFU
protocol~\cite{balakrishnan_corfu_2012}. The shared-log is a powerful
abstraction used to construct distributed systems, such as metadata
management~\cite{balakrishnan:sosp13} and elastic database
systems~\cite{bernstein:cidr11,bernstein:vldb11,bernstein:sigmod15}.  However,
existing implementations that rely on consensus algorithms such as Paxos funnel
I/O through a single point introducing a bottleneck that restricts throughput.
In contrast, the CORFU protocol is able to achieve high throughput using a
network counter called a \emph{sequencer}, that decouples log position
assignment from log I/O.

While a full description of the CORFU system is beyond the scope of this paper,
we briefly describe the custom storage device interface, sequencer service, and
recovery protocol, and how these services are instantiated in the Malacology
framework.

\subsubsection{Sequencer}
\label{sec:seq}

High-performance in CORFU is achieved using a sequencer service that assigns
log positions to clients by reading from a volatile, in-memory counter which
can run at a very high throughput and at low latency. Since the sequencer is
centralized, ensuring serialization in the common case is trivial.  The
primary challenge in CORFU is handling the failure of the sequencer in a way
that preserves correctness. Failure of the sequencer service in CORFU is
handled by a recovery algorithm that recomputes the new sequencer state using a
CORFU-specific custom storage interface to discover the tail of the log, while
simultaneously invalidating stale client requests using an epoch-based
protocol.

{\bf Sequencer interface.} The sequencer resource supports the ability to
\emph{read()} the current tail value and get the \emph{next()} position in the
log which also atomically increments the tail position.  We implement the
sequencer service in Malacology using the File Type interface.
This approach has the added benefit of allowing the metadata
service to handle naming, by representing each sequencer instance in the
standard POSIX hierarchical namespace. The primary challenge in mapping the
sequencer resource to the metadata service is handling serialization correctly
to maintain the global ordering provided by the CORFU protocol.

Initially we sought to directly model the sequencer service in Ceph as a
non-exclusive, non-cacheable resource, forcing clients to perform a round-trip
access to the resource at the authoritative metadata server for the sequencer
inode.  Interestingly, we found that the capability system in Ceph 
reduces metadata service load by allowing clients that open a shared
file to temporarily obtain an exclusive cached copy of the resource, resulting
in a round-robin, best-effort batching behavior. When a single client is
accessing the sequencer resource it is able to increment the sequencer locally.
Any competing client cannot query the sequencer until the metadata service has
granted it access.

While unexpected, this discovery allowed us to explore an implementation
strategy that we had not previously considered. In particular, for bursty
clients, and clients that can tolerate increased latency, this mode of
operation may allow a system to achieve much higher throughput than a system
with a centralized sequencer service.  We utilize the programmability of the
metadata service to define a new policy for handling capabilities that controls
the amount of time that clients are able to cache the sequencer resource. This
allows an administrator or application to control the trade-off between latency
and throughput beyond the standard best-effort policy that is present in Ceph
by default. 

In Section~\ref{sec:evaluation} we quantify the trade-offs of throughput and
latency for an approach based on a round-robin batching mode, and compare this
mode to one in which the metadata server mediates access to the sequencer state
when it is being shared among multiple clients. Quantifying these trade-offs
should provide administrators with guidelines for setting the tunables for
different ``caching" modes of the sequencer.

{\bf Balancing policies.} As opposed to the batching mode for controlling
access to the sequencer resource, more predictable latency can be achieved by
treating the sequencer inode as a shared non-cacheable resource, forcing
clients to make a round-trip to the metadata service. However, the shared
nature of the metadata service may prevent the sequencer from achieving maximum
throughput. To address this issue we use the Load Balancing interface to construct a
service-specific load balancing policy. As opposed to a balancing policy that
strives for uniform load distribution, a ZLog-specific policy may utilize
knowledge of inode types to migrate the sequencer service to provisioned
hardware during periods of contention or high demand.

\subsubsection{Storage Interface}

The storage interface is a critical component in the CORFU protocol. Clients
independently map log positions that they have obtained from the sequencer
service (described in detail in the next section) onto storage devices, while
storage devices provide an intelligent \emph{write}-once, random \emph{read}
interface for accessing log entries. The key to correctness in CORFU lies with
the enforcement of up-to-date epoch tags on client requests; requests tagged
with out-of-date epoch values are rejected, and clients are expected to request
a new tail from the sequencer after refreshing state from an auxiliary service.
This mechanism forms the basis for sequencer recovery.

In order to repopulate the sequencer state ({\it i.e.} the cached, current tail of
the log) during recovery of a sequencer, the maximum position in the log must
be obtained. To do this, the storage interface exposes an additional
\emph{seal} method that atomically installs a new epoch value and returns the
maximum log position that has been written.

Since the sequencer service does not resume until the recovery process has
completed, there cannot be a race with clients appending to the log, and the
immutability of the log allow reads to never block during a sequencer failure.
Recovery of the sequencer process itself may be handled in many ways, such as
leader election using an auxiliary service like Paxos. In our
implementation, the recovery is the same as (and is inherited from) the CephFS
metadata service. Handling the failure of a client that holds the sequencer
state is similar, although a timeout is used to determine when a client should
be considered unavailable.
