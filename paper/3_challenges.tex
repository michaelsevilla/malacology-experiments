\section{Challenges}
\label{sec:challenges}

Implementing the infrastructure for programmability into existing services and
abstractions of distributed storage systems is challenging, even if one assumes
that the source code of the storage system and the necessary expertise for
understanding it is available.  Some challenges include:

\begin{itemize}

\item Storage systems are generally required to be highly available so that any
complete restarts of the storage system to reprogram them is usually
unacceptable.

\item Policies and optimizations are usually hard-wired into the services and
one has to be careful when factoring them to avoid introducing additional bugs.
These policies and optimizations are usually cross-cutting solutions to
concerns or trade-offs that cannot be fully explored at the time the code is
written (as they relate to workload or hardware). Given these policies and
optimizations, decomposition of otherwise orthogonal internal abstractions can
be difficult or dangerous.

\item Mechanisms that are often only exercised according to hard-wired policies
and not in their full generality have hidden bugs that are revealed as soon as
those mechanisms are governed by different policies. In our experience
introducing programmability into a storage system proved to be a great
debugging tool.

\item Programmability, especially in live systems, implies changes that need to
be carefully managed by the system itself, including versioning and propagation
of those changes without affecting correctness.

\end{itemize}

To address these challenges we present Malacology,\oldcomment{. Malacology is
both a prototype for a programmable storage system and a} \newcomment{ our
prototype programmable storage system. It uses the programmable storage}
\addressesissue{2} design approach to evolve storage systems efficiently and
without jeopardizing years of code-hardening and performance optimization
efforts.  Although Malacology uses the internal abstractions of the underlying
storage system, including its subsystems, components, and implementations, we
emphasize that our system still addresses the general challenges outlined
above.

The main challenge of designing a programmable storage system is choosing the
right internal abstractions and picking the correct layers for exposing them.
\oldcomment{In this paper, we do not present an exhaustive list of possible
internal abstractions nor do we contend that the abstractions we choose provide
the best trade-offs for all applications.} \newcomment{A programmable storage
is not defined by what abstractions are exposed, rather a programmable storage
system adheres to the design approach of exposing interfaces so administrators
can have better control of the storage system. The interfaces presented in this
paper are abstractions that we found useful for building our prototype services
ZLog and Mantle, yet they may not provide the best trade-offs for all
higher-level services.} \addressesissue{2} For example, if consensus is
correctly exposed one could implement high-level features like versioning,
serialization, or various flavors of strongly consistent data management on
top; but perhaps a low-level consensus interface is suited well for a
particular set of applications.  These questions are not answered in this paper
and instead we focus on showing the feasibility of building such a system,
given advances in the quality and robustness of today's storage stacks.

The Malacology prototype we present has been implemented on Ceph.  While there
are other systems on top of which Malacology could be implemented (see
Table~\ref{table:examples}), we choose Ceph because it is a production quality
system and because it is open source. The large developer community ensures
that code is robust and the visibility of the code lets us expose any interface
we want. In the next section we describe the Ceph components that we expose as
Malacology interfaces.
